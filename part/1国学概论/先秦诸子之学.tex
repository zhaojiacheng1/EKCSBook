% !Tex root = ../../EKCSBook.tex
% 此处为chapter文件
\chapter{先秦诸子之学}

讲先秦诸子之学,有应知者数题:

1.诸子之学重在社会政治方面,不重于哲学科学方面,因诸子本身之发展及其对后来之影响皆如此(此意章炳麟曾言之)。\footnote{[诸子至于哲学方面,颇与古代希腊之哲学相近,其程度亦相仿。盖吾国古代原有此等与宗教混合之哲学思想。诸子即上承此等哲学,而并非加以发展,故诸子之哲学思想,大致相同,不若社会政治之学经发展进步而分歧也。与科学方面,亦有称述,而以见于《墨子》者最多。盖亦旧时之所有,墨子承之也。惟亦不重于此,故其后迄未有何发展。]}

2.古有专门\footnote{[专门者,以如今观之,实即一种学问之派别。]}而无通学,\footnote{[通学者,兼取各派,择善而从,至汉方有通学。]}故诸子之学,就一方面论之则精,合各方面论之则空。其相互攻驳之语,多昧于他人之立场,不合论理,如墨子贵俭,所欲行者乃古凶荒札丧之变礼。而荀子驳以“不足非天下之公患”,\footnote{[见《荀子·富国篇》]}殊不知墨子本不谓平世亦当如是也。\footnote{[古代治学寡,而因交通不便,得书不易,学术之传播亦难。学者仅能就其近者习之,远者不知或知之不详,且人具成见,学问常以先入为主,故当时人可与一种学问接触,终身不知其他者。此专门之学之所以成也。]}

3.先秦诸子之学,非皆个人创造,大抵前有所承,新旧适不适不等,盖其时间有早晚,又地域亦有开通与僻陋之别也。鄙意先秦诸子最要者六家,其新旧之别略如下:

最早者农家,沿袭简陋(时代或地域)之农家社会思想。次之者道家,代表简陋之游牧社会。次之者墨家,其思想与夏代政治颇有渊源。次之者儒家及阴阳家,见多识广,知若干种治法,应更迭使用。最新者法家,对外主张兼并,对内主张摧毁贵族,总而言之,是打倒封建势力(以开明专制为手段)。

农家之书尽亡,仅存者许行之说,见《孟子·滕文公》上篇。\footnote{[农家之书,真系讲树艺之术者,为《吕览》之《任地》《辨上》《审时》诸篇。然此非所重。先秦诸子皆欲以其道移易天下,非以百亩为己忧者也。《汉志》论农家之学云:“鄙者为之,欲使君臣并耕,悖上下之序。”可见《孟子》所载之许行,实为农家巨子。]}(1)谓贤君当与民“并耕而食,饔飧而治”。此犹乌丸大人,各自畜牧管产,不相徭役(见《后汉书》本传)。(2)主买卖论量不论质,此由交易不重要,物品本少使然,古盖自有此简陋之世;亦或战国尚有此等落伍之地。许行欲率天下而从之,则其事不可行矣。\footnote{[且复古必有其方,许行未尝有言(如其有之,则陈相当述之,孟子当驳之,不应徒就宗旨辩难),此则不能不令人疑其徒为高论者也。]}

道家之代表为《老子》,《老子》之旨在无为。为,化也。\footnote{[无为,犹言无化,古“为”“化”实为同字,观“譌”“訛”为同字之例可知。《论语》:“子曰:张而不弛,文、武弗能(耐)也。弛而不张,文、武弗为也。”此“为”字即“化”字义,言不能使谷物变化也。]}无化者,无使社会起变化。此犹今人慕效欧、美之文明,社会组织,因之改变。守旧者遂欲闭关绝市耳。当时落后之国,输入先进之国之文明者,盖(1)由其君大夫之好者,(2)由其自谓野蛮而欲驱其民以从当时所谓文明之俗,如商鞅谓秦初父子同室,吾今大筑翼阙,营如鲁、卫是也。\footnote{[古人恒以是为戒,如由余对秦穆公之言是也。]}《老子》最反对此等,故谓“无为而无不为”,犹言勿以汝之道化民,则民无不化而之善也。此说认社会之恶化,\footnote{[盖当时之效法文明,不过任其迁流所心,非有策划,改变社会之组织,以与之相应也。则物质文明日增,而社会组织随之坏矣。然道家不能改变社会组织,以与新文明相应,而徒欲阻遏文明,则何可得?]}皆由君大夫措施之误。而不知社会因日日在自化,\footnote{[盖人之趋势,如水就下。慕效文明,其利显而易见;社会组织变坏,其患隐而难知,且亦未必及己,人又孰肯念乱?故社会日日在自化也。]}老子特未之见也。

《庄子》历代著录,皆在道家,《管子》或属道或属法,二家之论,一部分诚与《老子》同。然讲个人在社会重自全之术而归结于委心任运,此《庄子》所有,而《老子》所无。\footnote{[《列子》说同《庄子》。盖其时代之晚,各个间互相之接触已多,世事变化无方,其祸福殊不可知,故有《齐物论》之说(论同伦,类也)。物论可齐,复何所羡?何所畏避?故主张委心任运。]}不思彻底改造,而只想因势利导(如不思去民好利之心,而徒欲因其好利而利用之),亦《管子》所有,而《老子》所无,此可见其时代之晚,其社会已不可控制,犹柏拉图与亚里斯多德之异也。

陈旧于农家道家者,为墨家。《淮南要略》云:“墨子学于孔子而不悦,背周道而用夏政。”\footnote{[今观《墨子》书,《修深》《亲士》《所染》纯为儒家言。他篇又多引《诗》《书》之文,则《淮南》之说是也。]}《吕氏春秋·当染》云:“鲁惠公使宰让请郊庙之礼于天子,天子使吏角往,惠公止之,其后在鲁,墨子学焉。”\footnote{[史固辨于明堂行政之典者。故墨子之学,诚为明堂之学也。]}古大庙大学,皆与明堂同物,前已言之。墨子最讲实用,而其书《经上下》《经说上下》《大取》《小取》六篇,讲哲学、伦理,兼及自然科学,极其清深者,古明堂为宗教哲学之所存也。然此非墨子宗旨所在,特师授以书,则从而传之耳。\footnote{[大学虽东周后尚不能尽废。然未闻有一人合,学成而出仕者。则以所肄者为宗教家言,非实用之事也。大学所教,既为宗教家言,故为涵养德性之地。《礼记》曰:“君子如欲化民成俗,其必由学乎?”又曰:“能为师,然后能长;能为长,然后能为君。师也者,所以学为君也。”又曰:“君子所不臣于其臣者二,当其为尸,则弗臣也;当其为师,则弗臣也。”乞言养老之礼,执酱而馈,执爵而醑(醑,虚口),所以隆重如此者,正以其所诣师者,其初乃教中尊宿耳。又《王制》曰:“出征执有罪,反释奠于学。”凯旋而释奠于学。由此二端,可见古代大学之性质,为宗教哲学之所存也。]}其宗旨所在,曰兼爱,而行之则以非攻。曰贵俭,而行之则以节用、节葬、非乐。所以动人者,曰天志(其天神为人格神),曰明鬼,而辅之以非命。曰上同,使下之人听于上。\footnote{[盖本夏道,而夏时较古,人之思虑较少,人与人对立程度浅,乐尽力以服从于其上也。]}曰上贤,盖前代亲亲,不如周人之甚。参观孙星衍《墨子后叙》,知用夏政之不虚也。

古书多以墨儒并称,亦以儒侠并称,侠者,后世江湖豪杰之流。盖封建制度之坏,士失所养。\footnote{[封建制度之诸侯、大夫,多喜养士,及其国灭家亡,或习奢侈而暇养士,而士失所养。]}而不能为农工商,乃别成为一阶级。性质近乎文者为儒,\footnote{[游说之士,大抵从儒中出。]}近乎武者为侠。孔子、墨子,乃就此两社会而感化之,并非两个阶级,为孔、墨所造成也。墨子长于守御(其书末二十篇),盖自侠之团体中来也(《墨子》非攻,故仅取兵法中守之一部分)。

儒为封建制度奔溃时失养之士,性质近乎文的阶级,前已言之(其性质见于《礼记》之《儒行》)。\footnote{[儒之义为柔,若曾子之竞竞自守,言必信,行必果者,盖其本来面目。]}孔子为此阶级中之闻人,惟孔子之道,不尽于儒。孔子之学颇博,多知前代之治法。此时前代治法之可考者,有夏、殷、周三代。孔子以为当更迭使用,于是有《春秋》通三统之义(谓封前二代之后以大国,使保存其治法,说见《春秋繁露》)。孔子又观治化升降,以为最古之时最美,谓大同,时代渐降则渐劣,谓小康,说见《礼记·礼运篇》。然则更劣于小康,必为乱世矣。《春秋》张三世之以,以二百四十年,分为三世,据乱而作,(表示治乱世之法。)进于升平,(小康)更进于太平,(大同)(见于《公羊》何《注》)盖欲逆挽世运,复于郅治也。

孔子之道,具于六经,而六经之中,《易》与《春秋》为尤要。《易》言原理,《春秋》据此原理而施诸人事。故曰:“《易》本隐以至显,《春秋》推见至隐。”(《史记》)其根据原理施诸人事,则恃君长为之。故《易》曰:“大哉乾元,万物资始,乃统天。”(《乾卦·系辞》)而“《春秋》以元之气,正天之端,以天之端,正王之政,以王之政,正诸侯之即位,以诸侯之即位,正四竟之治。”(《公羊》隐公元年《注》)此略近希腊柏拉图推最高之哲人为君之义。惟希腊人无一统思想,故只计及一国之君。孔子则不然,故又计及诸侯之上,当有一王耳。

孔子所谓大同,盖农业共产小社会。所谓小康,则封建之初期,阶级虽已成立,旧时共产社会之规模,尚未甚坏者也。自此以后,资本势力又继封建势力而起,治化只有日趋于劣。不知铲除阶级而欲借政治之力,以谋革命之彻底完成,可谓南辕北辙。然自近代以前,学者之见解,固皆如此(革命常为政治的),不足为怪。

《易》之大义,为“变易”“不易”“简易”三者。“变易”谓宇宙间一切现象,无一息而不变;“不易”谓万变之现象,仍有其不易之则。(如气候时时在变,四季亘古如斯。古人只有循环之思想,无进化之思想。辩证法之变动,非其所知)“简易”,则言自然力出于自然,非如人之作事,倦而必须休息,故能永不间断差忒,犹佛家之贵无为而为贱有为也。此意义亦甚周匝(《易》一名而含之义,见易纬《乾凿度》,《周易义疏·八论》之一引)。

儒家出于司徒之官,故重教化。而其教化也,必先之以养。孔子言先富后教(《论语·子路·子适卫章》),孟子言有恒产然后有恒心,首欲浚井田制度,继之以庠序之教(《梁惠王》上、《滕文公》上),此皆思想也。此为历代儒家之传统思想,将来当再言之。惟儒家在政治上之抱负,因社会组织已变,无由实施。其有于中国者,乃在社会方面:(1)重人与人相和亲,而不重政治力量之控制。(2)儒家最重中庸,故凡事不趋极端,制度风俗,皆不止积重难返,而中国人无顽固之病。(3)儒家重恕,“己所不欲,勿施于人”。\footnote{[谓“絜矩”。]}其标准极简单明了,而含义又极高深,所谓愚夫愚妇,与知与能,而圣人有所不能尽。恕成为普遍的人生哲学,无意间为社会保持公道,此儒家之大有造于中国社会者。

阴阳家之书尽亡,惟邹衍之说,略见《史记·孟荀列传》。其说看似荒诞,实则不过就空间(彼所谓中国)、时间(所谓黄帝以来)两方面,据所知者,求得其公例,而推诸未知者耳。其研究结果,盖以治国当有五种办法。更迭使用,是为五德终始,\footnote{[《汉书·严安传》载安上书引邹子之言:“政教文质者,所以云救也。当时则用,过则舍之,有易则易之。”即五德终始之说也。]}(衍之五德终始,始从所不胜,即水、土、木、金、火;汉末乃改从相生之次,为木、火、土、金、水)此正犹儒家之通三统,彼所谓一德,当有其一套治法,非如后世阴阳家专讲改正朔、易服色等空文也。(《汉志》有《邹奭子》十二篇,则已拟有实行之法,果难施与否,今不可知,要非如汉人之言五德者,徒以改正朔、易服色为尽其能事也)故与儒家可列为一阶段。\footnote{[《太史公自序》述其父谈之论,谓阴阳家言,“大祥而众忌讳,使人拘而多所畏”,此乃阴阳家之流失,而非其道遂尽于是也。]}

以上诸家知识,均得诸历史上,均欲效法前代,惟其所欲法者,新旧不同耳。惟法家则注重眼前的事实,\footnote{[切合于东周时势。]}故其立说最新。法家之“法”字,又有广狭二义,广义包法、术二者言之,狭义则与术相对,\footnote{[申不害言术,公孙鞅为法,韩非盖欲兼综二派者。]}法所以治民,术所以治治民之人也,见《韩子·定法篇》。法家之书,存者有三:(1)《管子》,\footnote{[二十四卷,原本八十六篇,今佚十篇。]}(2)《韩非子》,\footnote{[二十卷,五十五篇。]}(3)《商君书》\footnote{[五卷,原本二十九篇,今佚三篇幅。]}也。法术之论,(治民及驭臣下之术)三书多同。惟《管子》多官营大事业,干涉借贷,操纵商业之论(大体见《轻重》诸篇)。《商君书》则偏重一民于农战,(意欲遏抑商业)盖齐、秦经济发达程度不同,故其说如此。\footnote{[齐工商之业特盛,殷富殆冠海内;秦地广而腴,且有山林之利,开辟较晚,侈靡之风未甚。]}\footnote{[《韩非子》多言原理,兼及具体之条件。]}法家之论,能训练其民而用之;术家则能摧抑贵族,故用法家者多致富强,\footnote{[如韩申不害相韩昭侯十五年,内修政教,外应诸侯,终其身,无侵韩者。卫吴起为魏文侯将,拔秦五城,守西河以拒秦、韩,文侯卒,事其子武侯,遭谮奔楚,相楚悼王,南平百越,北并陈、蔡,却三晋,西伐秦,诸侯皆患楚之强。]}秦且以之并天下也。\footnote{[卫鞅(商鞅)入秦,说孝公变法修刑,内务耕稼,外劝战死之赏罚。孝公任之,遂大强。故秦并天下,原因虽有数端,以人事论,则能用法家之说,实为其一大端。]}盖惟用法家,乃能一民于农战,其兵强而且多(见《荀子·议兵》),亦惟用法家,故能进法术之士,而汰淫靡骄悍之贵族(列国皆贵族政治,独秦行官僚政治),政事乃克修举也。
