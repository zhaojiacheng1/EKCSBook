% !Tex root = ../../EKCSBook.tex
% 此处为chapter文件
\chapter{先秦诸子之渊源二——王官之学}

儒家,出于司徒之官。\footnote{[《汉书·艺文志》曰:“儒家者流,盖出于古司徒之官,助人君,顺阴阳,明教化。游文于六经之中,留义于仁义之际。祖述尧舜,宪章文武。宗师孔子,以重其言,于道最为高。”徒,众也,司徒主教化。《周礼》谓惟战事付司马,狱讼付司寇,此外治民之事,皆司徒掌之也。儒家治民,最重教化,此为其出司徒之官之本色,其欲合西周以前之法,斟酌而损益之。其处己之道,最高者为中庸。待人之道,最高者为絜矩。中庸者,随时随地,审处而求其至当;絜矩者,就所接之人,我所愿于彼者,即彼之所愿于我,而当以是先施之。]}

道家,出于史官。\footnote{[《汉书·艺文志》曰:“道家者流,盖出于古之史官。历记成败存亡祸福古今之道,然后知秉要执本,清虚以自守,卑弱以自持。此为君人南面之术。”其宗旨:一在守柔,一在无为,所称颂者,为黄帝时之说。]}

墨家,出于清庙之守。\footnote{[《汉书·艺文志》曰:“墨家者流,盖出于清庙之守。茅屋采椽,是以贵俭。养三老五更,是以兼爱。选士大射,是以尚贤。宗祀严父,是以右鬼。顺四时而行,是以非命。以孝视天下,是以上同。”盖古明堂、清庙、辟雍,皆一物也。蔡邕《明堂月令章句》谓:“明堂者,天子大庙,所以祭祀、飨功、养老、选士,皆在其中。取正室之貌,则曰大庙;取其正室,则曰大室;取其堂,则曰明堂;取其四时之学,则曰大学;取其圆水,则曰辟雍;虽名别实同。”(详见《续汉书·祭祀志注》)阮元《明堂说》谓:“有古之明堂,而有后世之明堂。古者政教朴略,宫室未兴,一切典礼,皆行于天子之居,后乃礼备而地分。礼不忘本,于近郊东南,别建明堂,以存古制。”(见所著《揅经室集》)盖古之清庙,原极简陋,墨家出于清庙之守,即欲以清庙之旧法,救当时之弊。其根本义曰兼爱,即所谓夏尚忠。其所欲行,盖夏道也。由兼爱故不容剥民自奉,而节用、节葬、非乐之说出。由兼爱故不容夺人所有,而非攻之论出。]}

名家,出于礼官。\footnote{[《汉书·艺文志》曰:“名家者流,盖出于古之礼官。古者名位不同,礼亦异数。孔子曰:‘必也正名乎?名不正,则言不顺;言不顺,则事不成。’”礼主差别,差别必有其由,深求其由,是为名家之学,督责之术;必求名实之相符,故与法家,关系殊密也。]}

法家,出于理官。\footnote{[《汉书·艺文志》曰:“法家者流,盖出于理官。信赏必罚,以辅礼制。”为切于东周时势之学。东周之要务有二:一为富国强兵,一为裁抑贵族。前者为法家言,后者为术家言,说见《韩非子·定法篇》。申不害言术,公孙鞅言法,韩非盖欲兼综二派者。法家宗旨,在“法自然”。故戒释法而任情。不主宽纵,亦不容失之严酷。]}

阴阳家,出于羲和之官(古之历法之官)。\footnote{[《汉书·艺文志》曰:“阴阳家者流,盖出于古羲和之官。敬顺昊天,历象日月星辰,敬授民时。”以邹衍为大师,本所已知,推所未知。其五德终始之说,亦犹儒家之有通三统之论。亦欲合西周之法,斟酌而损益之。]}

纵横家,出于行人之官。\footnote{[《汉书·艺文志》曰:“纵横家者流,盖出于古行人之官。当权事制宜,受命不受辞。”又曰:“及邪人为之,则上诈谖而弃其信。”则正指苏、张之流也。]}

农家,出于农稷之官。\footnote{[《汉书·艺文志》曰:“农家者流,盖出于古者农稷之官,播百谷,劝农桑,以足衣食。”《孟子》所载云许行,实为农家巨子,其言有二:一君臣并耕,一则物价但论多少,不论精粗也。盖皇古之俗。农家所愿,即在此神农以前之世也。]}

杂家,出于议官。\footnote{[《汉书·艺文志》曰:“杂家者流,出于议官。兼儒、墨,合名、法。知国体之有此,见王治之无不贯。”盖专门之学,往往蔽于其所不知。西汉以前,学多专门,实宜有以祛其弊。故但综合诸家,即可自成一学也。所谓议官,盖即《管子》所谓“啧室”(《管子·桓公问》:“黄帝立明台之议,尧有衢室之问,舜有告善之旌,禹立谏鼓于朝,汤有总街之庭,武王有灵台之复。欲立啧室之议,人有非上之过内焉”),而秦、汉之议郎(秦置,掌议论,汉特征贤良方正之士为之,秩比六百石,统于光禄勋。晋以后废),盖即古议官之制。而齐稷下谈士,四公子之养客,皆为此类。]}

小说家,出于稗官。\footnote{[《汉书·艺文志》曰:“小说家者流,盖出于稗官,街谈巷议,道听途说者之所造也。”疑《周官》诵训、训方氏之所采正此类。九流之说,皆出于士大夫,惟此人为人民所造。《汉志》所载,书已尽亡。《太平御览》卷八百六十六引《风俗通》,谓宋城门失火,汲池中水以沃之,鱼悉露见,但就取之。说出《百家》。犹可略见其面目也,他如塞翁失马、鲁酒薄而邯郸围等,亦或此类。]}

以上为《汉书·艺文志》诸子十家,其中取小说家,谓之九流,见《后汉书·张衡传》注\footnote{[《刘子·九流篇》同。]}《汉书·艺文志》本于刘向、歆父子《七略》,\footnote{[《汉书·艺文志》:“成帝时,诏刘向校经传诸子诗赋,向条其篇目,撮其旨意,录而奏之。会向卒,向子歆总群书,而奏《七略》。故有《辑略》《六艺略》《诸子略》《诗赋略》《兵书略》《术数略》《方技略》。”]}乃据汉时王室藏书而为之分类,故于学术流别,最为完全。古平民无学术,\footnote{[王官者,大国之机关也。诸子出王官说,虽为汉人推论,然极有理,当时平民,无研究学术者。虽有学术思想,有志研究,亦无所承受,无所商讨,即有所得,亦无人承继之。而古代学术,为贵族所专有,然贵族亦非积有根柢,不能有所成就。王官专理一业,守之以世,岁月既久,经验自宏,其能有所成就,亦固其所。]}近人胡适据《淮南要略》作《九流不出王官论》,\footnote{[载《新青年》杂志],约当民国四、五、六年时。}以驳《汉志》,殊不知《汉志》言其由来,《淮南》言其促进之动机(所谓救时之弊)。\footnote{[盖王官之学,固颇有成就,然非遭世变,乡学者不得如此其多,即其所成就,亦不得如此之大也。故《汉志》言因,《淮南》言缘也。]}二者各不相妨,且互相补足也。\footnote{[若谓出于王官之说非,而惟本《淮南》之说。则试观诸子之内容、文辞,多今古间杂,明非一世之物,惟其源本王官,故能多本往事以立说也。]}
