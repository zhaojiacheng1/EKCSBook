% !Tex root = ../../EKCSBook.tex
% 此处为chapter文件
\chapter{先秦诸子之渊源一——古代之宗教哲学}

宗教哲学,今日为对立之物,在宗教起源之时则不然。一种宗教之初兴,必能综合当时人之宇宙及人生观,而为之谋得一合理之解决。此时之宗教,亦即其时最伟大最适宜得哲学。\footnote{[凡一种大宗教,必具高深及浅显二方面之理论,以满足于高等与低等之人。]}但宗教之为物,不徒重理智,而兼重意志及 情感,故易于固执具体得条件,久之,遂变为落伍之物,而哲学乃与之分离。\footnote{[宗教与理智方面,仅为其一种手段,使人得理智上之满足,其注重者,乃偏于意志及感情方面,使信之者得感情之安慰,秉坚强之意志以信仰其宗教。夫意志坚强之人,固不免易于固执也,固信教者,常以强烈之感情徘徊,以坚强之意志守旧,以致久而落伍。]}故宗教之与哲学对立,非其本来对立,而由于宗教之陈旧。而论古代之学术,仍必溯源于宗教。

中国之宗教思想,最早盖系“拜物”,《周礼》分祭祀之对象为天神、地祇、人鬼、物魅四类。物魅盖即拜物时代之遗也。此时之思想,太觉幼稚,对于学术思想,无甚影响。

稍进,则为“天象崇拜”,其中又分二期,前期盖在女权昌盛时代,所崇拜之大神,\footnote{[神之为物,表面视为与人无关的,人外的,实则此种之神,已成为人为的神,具有人之性质。神之组织及一切,皆以人为依据,与人相类。]}悉视为女性。《礼记·郊特性》曰:“郊之祭也,大报天而主日。”可见古无抽象的整个的天神。后世祭天之郊祭,只是祭日,而《楚辞》《山海经》皆以羲和为女神。整个的地神,古代更其没有,所祭的只是自己所住所种的一块土地,是为社祭。地神的被视为女性,更古今皆然。关于此问题,可参看日本田崎仁义所著《中国古代经济思想及制度》,王学文译,商务印书馆本。

女权时代之思想,存留于世者甚少,只有一部《老子》,是表现女性优胜思想的。《礼记·礼运》,孔子曰:“我欲观殷礼,是故之宋,而不足征也。吾得乾坤焉。”说者谓殷《易》先坤。这该是一种女性优胜得遗迹,但殷《易》之内容不传,今所传之《周易》,则完全表现男性优胜之思想矣。\footnote{[《周易》先乾。]}

大概中国学术思想,大部分是从周朝流传下来的,殷以前的成分,已经很少了,周朝宗法特别发达,可见其为男权昌盛的社会。而殷朝兄终弟及,是一个母权社会的遗迹。周在西方,殷在东方。后来齐国长女,名为巫儿,为家主祠,不出嫁\footnote{[《汉书·地理志》]},而齐太公为出夫\footnote{[《战国策》]}。燕人宾客相遇,以妇侍宿,婚嫁之夕,男女无别,反以为荣\footnote{[亦见《汉书·地理志》]}。楚王妻妹\footnote{[《公羊》恒公二年]}。皆可见风俗与周不同。然此等文化,多已沦亡了。

古代宗教哲学之骨干,为阴阳五行,但二者似非一说。\footnote{[阴阳最早见于《周易》,然不及五行之说,五行最早见于《洪范》,然不及阴阳之说。且自古至今,从未有合论二者者,后世言阴阳者众,而说五行者寡,则以阴阳能自圆其说,而五行不能也]}五行之说,见于《书》之《洪范》,后世衍其说最详者,为《白虎通义》之《五行篇》,虽煞费苦心,然究属勉强,予意水、火、木、金、土乃古代管此工事之五种官。\footnote{[水,通沟渠,建桥梁。火,未知钻木取火,得火不易,设之以保存火种。木,伐木制物。金,冶金之事。土,营建之事。]}本非哲学上分物质此为五类,后来哲学家就已成事实而强为之说。\footnote{[物体之之然,气、液、固体。印度之哲学,言地、水、风、火四大。地,固体;水,液体;风,气体,益之以火。如此分法,尚觉可通。而五行之水、火、木、金、土,水为液体,木、金、土为固体占其三,无一气体,不必哲学家,亦知为不妥,故决非哲学家分物质为此五类。]}其说本来不合论理,故虽煞费苦心,终不能自圆其说也。\footnote{[五行之变化,有生胜之说,亦作生克,水生木,木生火,火生土,土生金,金生水,水克火,火克金,金克木,木克土,土克水。其中火生土、土生金、金生水、木克土、土克水等,实不甚可通。]}

至于阴阳,则因人之认识,必始于两,而现象无论如何错杂,亦总可归纳之而成二组,即谋与非谋,此即正负,正负即阴阳也。故其说处处可通,而古人推论万物,必自小而推诸大,于是以天地为万物之父母。\footnote{[此时以阴阳为实质,尚未合哲学之原理。]}再进,知万物之原质推一,乃名此原质曰气。假想万物之变化,皆由于气之聚散,而气之所以动荡不已,则由于阴阳二力(阴静阳动)之更迭起伏焉。至此而哲学上之泛神论成立矣。\footnote{[泛神论者,即谓神即宇宙间一切现象之本身。]}

宗教哲学之进步,既进于泛神论,则事物变动之原因,即在事物之本身。(古代野蛮人不知自然规律,只知人律,其视万物为有知,一切皆神所为。而其所谓神,亦有实体,《墨子·天志》《明鬼》之论,所谓天、鬼者,皆有喜怒,欲恶如人,则其证也。遂泛神论既成立,至此自有神而进为无神矣)而别无一物焉,在其外而使之者,此之谓自然。\footnote{[自,始也;然,成也。古书自然之“然”字,无作如此之解者]}自然之力,至为伟大,只有随顺利用,而不能抵挡。而自然界之美德,如“不息”“有秩序”“不差忒”等,均为人所宜效法,此之谓“法自然”。\footnote{[法家即如是,谓政治上之赏罚,当以自然界之美德为准则也。]}自然之规律,道家称之曰道。\footnote{[《老子》曰:“有物混成,先天地生,寂兮寥兮,独立而不改,周行而不殆,可以为天下母,吾不知其名,字之曰道。”]}自然现象,永远变动不居,而其变动也,又有一定之规律,是为“变易”,“不易”,加以永不止息,若人之任事,不觉其劳苦然,是为“简易”。所谓“易一名而兼三义”也,不易之现象,是为循环,祸福倚伏之义,由此而生。\footnote{[吾国古代以农为主,注意于不易之现象,故有祸福倚伏等义。盖以为自然界之现象既如是,人事亦当如是也。]}自然力既伟大,人根本绝无能为,委心任运之义,由此而生。

自然之力,固无从从时间上指其原理,亦无从从空间上指其根源。然强为之名,固可如此。\footnote{[盖泛神论者,本无因果也,惟无因果之关系,则无可加以思想,故强为之名,而为是说者,亦明知其为强说而已。]}此种力之原始(系人所强名的),儒家称之曰“元”。《易》曰:“大哉乾元,万物资始,乃统天。”(《乾卦·彖辞》)《春秋》家谓:“《春秋》以元之气,正天之端。”(《公羊》隐公元年何注)是也。\footnote{[其他如《公羊》何注,隐公元年,曰:“天不深正其元,则不能成化。”《春秋繁露·重政》曰:“元者,万物之本,在乎天地之前。”]}(古人以天为万物之原因,而元为天之原因)在男权优胜时代,最贵此种健行的美德,但女权优胜时代柔能克刚、静以制动等见解\footnote{[今所存者,以《老子》为代表。]}亦仍保有相当的地位。
