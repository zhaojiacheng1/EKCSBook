% !Tex root = ../../EKCSBook.tex
% 此处为chapter文件
\chapter{秦汉时代学术之新趋势}

1.交通便利,各种学术,渐相接触,启通学之机。\footnote{[前此列国互相猜忌,往来之间,非有节符不能通,而关之稽查尤严,其极,遂至借以为暴。汉有天下,此弊尽去,有“通关梁,一符传”之美谈,而交通便利矣。]}2.利用学术者,将抉择或折衷于诸学之间,而求其至当。故其结果,为不适宜于时代之学术,渐见衰息(如墨家、农家)。其适宜者亦渐与他学相混焉。\footnote{[至此自先秦之新发明的时代,变为两汉之整理的时代。犹西洋史上之希腊———发明——与罗马——整理——之关系也。]}

秦有天下,仍守法家之学不变,然此时法家用整齐严酷之法,以训练其民之办法,实已用不着。\footnote{[法家宗旨,在“法自然”,故戒释法而任情。揆其意,固不主于宽纵,亦不容失之严酷。然专欲富国强兵,终不免以人为殉。《韩非子·备内篇》云:“王良爱马,为其可以驰驱;勾践爱人,乃欲用以战斗。”情见乎辞矣。在列国相争,争求统一之时,可以暂用,治平一统之时而犹用之,则恋蘧庐而不舍矣。秦之速亡,亦不得谓非过用法家言之咎。]}秦亡之后,众皆以其刻薄寡恩,归咎于法家之学,一时遂为众所忌讳。是时急于休养生息,故道家之说颇行。\footnote{[如孝惠元年,曹参相齐,尊治黄、老言者盖公,为言“道贵清静,而民自定”,参用之,相齐九年,齐国安集。及继萧何为相,举事无所变更,择谨厚长者为郡国吏,掩人细过,不事事,百姓歌之,有“载其清靖,民以定壹”之辞。孝景时,窦太后好黄、老术,皆其著者。]}然道家主无为,为正常之社会言之则可;社会已不正常,而犹言无为,是有病而不治也。故其说亦不能大盛。
